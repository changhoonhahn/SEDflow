\section{Summary and Outlook} \label{sec:summary}
By analyzing the SED of a galaxy, we can infer detailed physical properties
such as its stellar mass, star formation rate, metallicity, and dust content. 
These properties serve as the building blocks of our understanding of how
galaxies form and evolve. 
State-of-the-art SED modeling methods use MCMC sampling to perform Bayesian
statistical inference. 
They derive posterior probability distributions of galaxy properties given
observation that accurately estimate uncertainties and parameter degeneracies
to enable more rigorous statistical analyses. 
%Posteriors also enable marginalization over any nuisance parameters. 
For the dimensionality of current SED models, deriving a posterior requires 
${\gtrsim}100,000$ model evaluations and take ${\gtrsim}10-100$ CPU hours per 
galaxy. 
Upcoming galaxy surveys, however, will observe \emph{billions} of galaxies
using \emph{e.g.} DESI, PFS, Rubin observatory, James Webb Space Telescope, and
the Roman Space Telescope. 
Analyzing all of these galaxies with current Bayesian SED models is infeasible
and would require hundreds of billions of CPU hours.
Even with recently proposed emulators, which accelerate model evaluations by
three to four orders to magnitude, the computation cost of SED modeling would
remain a major bottleneck for galaxy studies. 

We demonstrate in this work that Amortized Neural Posterior Estimation (ANPE)
provides an alternative \emph{scalable} approach for Bayesian inference in SED
modeling.
ANPE is a simulation-based inference method that formulates Bayesian inference
as a density estimation problem and uses neural density estimators (NDE) to
approximate the posterior over the full space of observations. 
The NDE is trained using parameter values drawn from the prior and mock
observations simulated with these parameters.  
Once trained, a posterior can be obtained from the NDE by providing the
observations as the conditional variables without any additional model
evaluations. 

In this work, we present {\sc SEDflow}, a galaxy SED modeling method using ANPE
and PROVABGS, a flexible SED model that uses a compact non-parameteric SFH and
ZH prescriptions and was recently validated in \cite{hahn2022}.
Furthermore, we apply {\sc SEDflow} to optical photometry from the NASA-Sloan
Atlas as demonstration and validation of our ANPE approach.  
We present the following key results from our analysis.
\begin{compactitem}
    \item We train {\sc SEDflow} using a data set of ${\sim}1$ million SED
        model parameters and forward model synthetic SEDs.
        The parameters are drawn from a prior and the forward model is based on
        the PROVABGS and noise models. 
        We design the ANPE to estimate $p(\btheta | f_X, \sigma_X, z)$, where
        $f_X$, $\sigma_X$, and $z$ are the photometry, photometric uncertainty,
        and redshift, respectively. 
        For its architecture, we use a MAF normalizing flow with 15 MADE blocks
        each with 2 hidden layers and 500 hidden units.
        Training {\sc SEDflow} requires roughly 1 day on a single CPU. 
        Once trained, deriving posteriors of galaxy properties for a galaxy
        takes ${\sim}1$ second, $10^5\times$ faster than traditional MCMC sampling. 
    \item Posteriors derived using {\sc SEDflow} show excellent agreement with
        posteriors derived from MCMC sampling. 
        We further validate the accuracy of the posteriors by applying  {\sc
        SEDflow} to synthetic observations with known true parameter values.  
        Based on statistical metrics used in the literature (p-p plot and SBC),
        we find excellent agreement between the {\sc SEDflow} and the true
        posteriors. 
    \item Lastly, we demonstrate the advantages of {\sc SEDflow} by applying it
        to the NASA-Sloan Atlas.
        Estimating the posterior of ${\sim}33,000$ galaxies takes <12 CPU hours. 
        We make the catalog of posteriors publicly available at \peter{URL}. 
        For each galaxy, the catalog contains posteriors on all 12 PROVABGS
        SED model parameters.
        In terms of galaxy properties, the catalog includes posteriors of
        $M_*$, average SFR over 1Gyr, mass-weighted metallicity, mass-weighted
        age, and dust optical depth. 
\end{compactitem}

This work highlights the advantages of using an ANPE approach to Bayesian SED
modeling. 
Our approach can easily be extended beyond this application. 
For instance, we can include multi-wavelength photometry at ultra-violet or
infrared wavelengths. 
We can also modify \sedflow~to infer redshift from photometry. 
In \sedflow, we include redshift as a conditional variable, since NSA provides
spectroscopic redshifts. 
However, redshift can be included as an inferred variable rather than a
conditional one. 
Then, we can apply \sedflow~to infer galaxy properties from photometric data
sets without redshift measurements while marginalizing over the redshift
prior. 
If we do not require spectroscopic redshifts, \sedflow~can be extended to much
larger data sets that span fainter and broader galaxy samples. 
Conversely, we can use \sedflow~to infer more physically motivated photometric 
redshifts, where we marginalize over our understanding of galaxies rather than
using templates. 

The ANPE approach to SED modeling can also be extended to galaxy spectra. 
Constructing an ANPE for the full data space of spectra, however, requires
estimating a dramatically higher dimensional probability distribution. 
SDSS spectra, for instance, have ${\sim}3,600$ spectral elements.  
Furthermore, in our approach we include the uncertainties of observables as
conditional variables, which double the curse of dimensionality.
Recent works, however, have demonstrated that galaxy spectra can be represented
in a compact low-dimensional space using autoencoders~\citep[][Melchior \&
Hahn, in prep.]{portillo2020}.
In \cite{portillo2020}, they demonstrate that SDSS galaxy spectra can be
compressed into 7-dimensional latent variable space with little loss of
information. 
Such spectral compression dramatically reduces the dimensionality of the
conditional variable space to dimensions that can be tackled by current ANPE
methods. 
We will explore SED modeling of galaxy spectrophotometry using ANPE and
spectral compression in a following work. 
