\section{Introduction} \label{sec:intro} 
Physical properties of galaxies are the building
blocks of our understanding of galaxies and their evolution. 
We can determine properties such as stellar mass ($M_*$), star formation rate (SFR), metallicity
($Z$), and age ($t_{\rm age}$) of a galaxy by analyzing its
spectral energy distribution (SED).
Theoretical modeling of galaxy SEDs is currently based on stellar population
synthesis (SPS) and describes the SED as a composite stellar population
constructed
from isochrones, stellar spectra, an initial mass function (IMF), a star
formation and chemical evolution history, and dust
attenuation~\citep[\emph{e.g.}][see \citealt{walcher2011, conroy2013} for a
comprehensive review]{bruzual2003, maraston2005, conroy2009}.
Some models also include dust and nebular emissions as well as emissions from
active galactic nuclei~\citep[\emph{e.g.}][]{johnson2021}.
In state-of-the-art SED modeling, theoretical SPS models are compared to
observed SEDs using Bayesian inference, which accurately quantifies parameter
uncertainties and degeneracies among them~\citep{acquaviva2011,
chevallard2016, leja2017, carnall2018, johnson2021, hahn2022}. 
The Bayesian approach also enables marginalization over nuisance parameters,
which are necessary to model the effects of observational systematics
(\emph{e.g.} flux calibration).

However, current Bayesian SED modeling methods, which use Markov Chain Monte
Carlo (MCMC) sampling techniques, take $10-100$ CPU hours per
galaxy~\citep[\emph{e.g.}][]{carnall2019a, tacchella2021}. 
While this is merely very expensive with current data sets of hundreds of
thousands of galaxy SEDs, observed by the Sloan Digital Sky
Survey~\citep[SDSS;][]{york2000}, DEEP2~\citep{davis2003},
COSMOS~\citep{scoville2007}, and GAMA~\citep{baldry2018}, it is prohibitive for
the next generation of surveys.
Over the next decade, surveys with the 
Dark Energy Spectroscopic Instrument~\citep[DESI;][]{desicollaboration2016},
the Prime Focus Spectrograph~\citep[PFS;][]{takada2014}, 
the Vera C. Rubin Observatory~\citep{ivezic2019}, 
the James Webb Space Telescope~\citep{gardner2006},
and the Roman Space Telescope~\citep{spergel2015}, will observe \emph{billions}
of galaxy SEDs.
The task of SED modeling alone for these surveys would amount to tens or
hundreds of billions of CPU hours, exceeding \emph{e.g.} the entire compute
allocation of the Legacy Survey of Space and Time (LSST) data release
production\footnote{$\approx$2 billion core hours
(\url{https://dmtn-135.lsst.io/})} by at least two orders of magnitude.
Recently, \cite{alsing2020} adopted neural emulators to accelerate SED model
evaluations by three to four orders of magnitude --- posterior inference takes
minutes per galaxy.
While this renders current data sets within reach, the next generation data
sets will still require tens or hundreds of millions of CPU hours whenever any
aspect of the SED model is altered.
Furthermore, this still practically precludes rapid analyses of upcoming
transient surveys, especially LSST, which will report $\sim$10,000 alerts per
minute\footnote{\url{https://dmtn-102.lsst.io/}}.

But Bayesian inference does not require MCMC sampling.  
Simulation-based inference (SBI) is a rapidly developing class of inference
methods that offers alternatives for many applications~\citep[see][and
references therein]{cranmer2020}.
Many SBI methods leverage the latest developments in statistics and Machine
Learning for more efficient posterior estimation~\citep{papamakarios2017,
alsing2019a, hahn2019c, dax2021, huppenkothen2021, zhang2021}. 
Of particular interest for SED modeling is a technique called Amortized
Neural Posterior Estimation (ANPE). 
Instead of using MCMC to sample the posterior for every single galaxy
separately, ANPE uses neural density estimators (NDE) to build a model of the
posterior for \emph{all} observable galaxies.
Once the NDE is trained, generating the posterior requires only the observed
SED and no additional model evaluations.

In this work, we present \sedflow, a method that applies ANPE to Bayesian
galaxy SED modeling using the recent \cite{hahn2022} SED model. 
We demonstrate that we can derive accurate posteriors with \sedflow~and make
Bayesian SED modeling fully scalable for the billions of galaxies that will be
observed by upcoming surveys.
As further demonstration, we apply \sedflow~to the optical photometry of
${\sim}33,000$ galaxies in the NASA-Sloan Atlas~(NSA). 
We begin in Section~\ref{sec:sbi} by describing SBI using ANPE.
We then present how we design and train \sedflow~in Section~\ref{sec:sedflow}
and describe the NSA observations in Section~\ref{sec:obs}. 
We validate the accuracy of the posteriors from \sedflow~in
Section~\ref{sec:results} and discuss the implications of our results in
Section~\ref{sec:discuss}. 
