\section{Summary} \label{sec:summary}
By analyzing the SED of a galaxy, we can infer detailed physical properties
such as its stellar mass, star formation rate, metallicity, and dust content. 
These properties serve as the building blocks of our understanding of how
galaxies form and evolve. 
State-of-the-art SED modeling methods use MCMC sampling to perform Bayesian
statistical inference. 
They derive posterior probability distributions of galaxy properties given
observation that accurately estimate uncertainties and parameter degeneracies
to enable more rigorous statistical analyses. 
%Posteriors also enable marginalization over any nuisance parameters. 
For the dimensionality of current SED models, deriving a posterior requires 
${\gtrsim}100,000$ model evaluations and take ${\gtrsim}10-100$ CPU hours per 
galaxy. 
Upcoming galaxy surveys, however, will observe \emph{millions} of galaxies
using \emph{e.g.} DESI, PFS, Rubin observatory, James Webb Space Telescope, and
the Roman Space Telescope. 
Analyzing all of these galaxies with current Bayesian SED models is infeasible
and would require \emph{billions} of CPU hours. 
%Furthermore, computational costs will only multiply as SED models increase in sophistication and treat theoretical uncertainties more flexibly. 
Rigorous inference would be the major bottleneck for galaxy studies. 

We demonstrate in this work that Amortized Neural Posterior Estimation (ANPE)
provides an alternative \emph{scalable} approach for Bayesian inference in SED
modeling. 
ANPE is a simulation-based inference method that leverages the latest
developments in ML.
It formulates Bayesian inference into a density estimation problem and uses 
neural density estimators (NDE) to approximate the posterior over the full
space of observations. 
The NDE is trained using parameter values drawn from the prior and mock
observations forward modeled using them.  
Once trained, a posterior can be derived using the NDE by plugging in the observations as the conditional variables without any additional model evaluations. 

In this work, we present {\sc SEDflow}, a galaxy SED modeling method using ANPE
and PROVABGS, a flexible SED model that uses a compact non-parameteric SFH and
ZH prescriptions and was recently validated in \cite{hahn2022}.
Furthermore, we apply {\sc SEDflow} to optical photometry from the NASA-Sloan
Atlas as demonstration and validation of our ANPE approach.  
We present the following key results from our analysis.
\begin{itemize}
    \item We train {\sc SEDflow} using a data set of ${\sim}1$ million SED
        model parameters and forward model synthetic SEDs.
        The parameters are drawn from a prior and the forward model is based on
        the PROVABGS and noise models. 
        We design the ANPE to estimate $p(\theta | f_X, \sigma_X, z)$, where
        $f_X$, $\sigma_X$, and $z$ are the photometry, photometric uncertainty,
        and redshift respectively. 
        For its architecture, we use a MAF normalizing flow with 15 MADE blocks
        each with 2 hidden layers and 500 hidden units.
        Training {\sc SEDflow} requires roughly 1 day on a single CPU. 
        Once trained, deriving posteriors of galaxy properties for a galaxy
        takes ${\sim}1$ second --- $>10^4\times$ faster than current 
        methods.
    \item Posteriors derived using {\sc SEDflow} show excellent agreement with
        posteriors derived from MCMC sampling. 
        We further validate the accuracy of its posteriors by applying  {\sc
        SEDflow} to synthetic observations with known true parameter values.  
        Based on statistical metrics used in the literature (p-p plot and SBC),
        we find excellent agreement between the {\sc SEDflow} and the true
        posterior. 
    \item Lastly, we demonstrate the advantages of {\sc SEDflow} by applying it
        to the NASA-Sloan Atlas.
        Estimating the posterior of ${\sim}34,000$ galaxies takes \todo{X} 
        CPU hours. 
        We make the catalog of posteriors publicly available at \url{urlhere}. 
        For each galaxy, the catalog contains posteriors on all 12 PROVABGS
        SED model parameters.
        In terms of galaxy properties, the catalog includes posteriors of
        $M_*$, average SFR over 1Gyr, mass-weighted metallicity, mass-weighted
        age, and dust optical depth. 
\end{itemize}

This work highlights the advantages of using an ANPE approach to Bayesian SED
modeling.  
Our approach will enable us to analyze the billions of galaxies that will be
observed by upcoming experiments and accurately measure their physical
properties. 
In addition, the posteriors on galaxy properties for all these galaxies we will
be able to construct probabilistic galaxy catalogs that will unlock a new level
of statistical robustness in galaxy studies and more fully extract the
statistical power of future observations.

% paragraph summarizing discussions and mentino transients 
Furthermore, since ANPE effectively removes the computational bottleneck of SED
modeling, it will enable us to address other aspects of SED modeling such as
the impact of model priors or limitations in SPS. 
\sedflow~can also be applied to transient science with upcoming experiments. 
For instance, host galaxies properties can be derived seconds after the event.
More broadly, the ANPE approach will unlock sophisticated  scalable 


In this work, we focus on SED modeling of optical photometry. 
The performance of ANPE has already been demonstrated on higher dimensional
applications. 
This means that \sedflow~can easily be extended to multi-wavelength photometry.
In fact, with the latest developments in efficiently dimensionality reduction
of spectra~\citep[][Melchior \& Hahn 2022]{portillo2020}, \sedflow~can even be
extended to galaxy spectra. 
We wil explore this in subsequent work. 
