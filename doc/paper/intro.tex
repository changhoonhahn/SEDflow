\section{Introduction} \label{sec:intro} 
Over the past few decades, observations from large galaxy surveys such as the
Sloan Digital Sky Survey~\citep[SDSS;][]{york2000}, DEEP2~\citep{davis2003},
COSMOS and zCOSMOS~\citep{scoville2007, lilly2007}, and GAMA~\citep{baldry2018}
have transformed our understanding of how galaxies form and evolve. 
The building blocks of our physical insight into galaxies are their 
physical properties, \emph{e.g.} stellar mass ($M_*$), star formation rate
(SFR), and metallicity ($Z$), measured from observations.
The primary way for doing this is by analyzing their spectral energy
distribution (SED).

% primer on SED modeling 
All of the physical processes in a galaxy leave an imprint on its SED:
\emph{e.g.} its star formation history, chemical enrichment history, dust
content. 
For instance, the SED over the ultraviolet to infrared wavelengths primarily
comes from the galaxy's stellar populations. 
Some of this stellar light is reprocessed by the gas and dust in its
interstellar medium.
The goal of SED modeling is to extract the detailed physical properties of
galaxies that are encoded in observed SEDs. 
SED modeling involves three key components: the observations, a physical SED
model, and a statistical inference framework for deriving physical properties 
from the comparisons between the observations and SED models.

% brief primer on SPS 
Current SED models are based on stellar population synthesis (SPS).
Broadly speaking, they model the SED of a galaxy as a compsite stellar
population constructed based on isocrhones, stellar spectra, an initial mass
function (IMF), a star formation and chemical evolution history, and dust
attenuation~\citep[\emph{e.g.}][]{bruzual2003, maraston2005, conroy2009,
eldridge2017}.
Some models further include dust and nebular emissions as well as emissions
from active galactic nuclei (AGN)~\citep[\emph{e.g.}][]{johnson2021}.
For a comprehensive review on SPS and SED modeling, we refer readers to
\cite{walcher2011} and \cite{conroy2013}. 
%The physical models are typically chosen to be stellar population synthesis codes, which combine  dust and nebular emission to create the SEDs of complex stellar popu- lations. There are many publicly available stellar pop-

In this work, we focus on the statistical framework used by current methods to
infer the galaxy properties from observations. 

State-of-the-art SED modeling 
For the statistical framework 

Bayesian parameter
inference framework to statistically infer galaxy properties from observations.
\todo{why is the Bayesian inference good:
allows informative priors based on prior observations,etc.}
Furthermore, the Bayesian framework allows us to marginalize over nuisance
parameters and accurately capture degeneracies among model parameters. 
To infer posteriors of high dimensional parameter space of SED model
parameters, state-of-the-art methods use MCMC or HMC to efficiently sample a
high dimensional parameter space. 
\todo{Go through all the latest works and detail their dimensionality and
setups.}

Despite these advantages, the current Bayesian SED modeling methods have a
major limitation: they are not scalable. 
SED modeling takes hundreds of hour per galaxy. 
For example in \cite{}, analyzing $\sim 4000$ galaxies in the LEGA-C ESO
Public Spectroscopic Survey~\cite{} using {\sc Bagpipes}required 3.5 million
CPU hours.
\todo{another example here}.
This limitation will only become more restrictive. 
SED models will become more sophisticated both in terms of stellar population
synthesis modeling and accounting for observational effects.  
The parameter space of SED models will only increase and make sampling
techniques less efficient. 
Moreover, upcoming surveys will increasing galaxy observations by
orders-of-magnitude.
The Dark Energy Spectroscopic Survey~\citep[DESI;][]{desicollaboration2016},
the Prime Focus Spectrograph~\citep[PFS;][]{takada2014},
Rubin~\chedit{CITATION}, and Roman~\chedit{CITATION} will observe SEDs of
millions of galaxies. 
Without a way to scale Bayesian SED modeling to millions of galaxies, the field
is left with doing shitty analyses or blowing huge computational resourcse. 

Simulation-based inference (SBI) using Amortized Neural Posterior Estimation
(ANPE) provides a life line. 
With ANPE, we can conduct rigorous Bayesian inference using only a fraction of
the computational resoruces of MCMC sampling techniques.
\todo{provide an overview of simulation based inference}. 
\todo{list examples}
In this work, we apply SBI using ANPE to Bayesian galaxy SED modeling. 
We demonstrate that with ANPE, we can make Bayesian SED modeling fully scalable
for the millions of galaxies that will be observed by upcoming surveys.

We begin in Section~\ref{sec:}



