\documentclass[12pt, letterpaper, preprint, comicneue]{aastex63}
%\usepackage[default]{comicneue} % comic sans font for editing
\usepackage[T1]{fontenc}
\usepackage{color}
\usepackage{amsmath}
\usepackage{natbib}
\usepackage{ctable}
\usepackage{bm}
\usepackage[normalem]{ulem} % Added by MS for \sout -> not required for final version
\usepackage{xspace}

% typesetting shih
\linespread{1.08} % close to 10/13 spacing
\setlength{\parindent}{1.08\baselineskip} % Bringhurst
\setlength{\parskip}{0ex}
\let\oldbibliography\thebibliography % killin' me.
\renewcommand{\thebibliography}[1]{%
  \oldbibliography{#1}%
  \setlength{\itemsep}{0pt}%
  \setlength{\parsep}{0pt}%
  \setlength{\parskip}{0pt}%
  \setlength{\bibsep}{0ex}
  \raggedright
}
\setlength{\footnotesep}{0ex} % seriously?

% citation alias

% math shih
\newcommand{\setof}[1]{\left\{{#1}\right\}}
\newcommand{\given}{\,|\,}
\newcommand{\lss}{{\small{LSS}}\xspace}

\newcommand{\Om}{\Omega_{\rm m}} 
\newcommand{\Ob}{\Omega_{\rm b}} 
\newcommand{\OL}{\Omega_\Lambda}
\newcommand{\smnu}{M_\nu}
\newcommand{\sig}{\sigma_8} 
\newcommand{\mmin}{M_{\rm min}}
\newcommand{\BOk}{\widehat{B}_0} 
\newcommand{\hmpc}{\,h/\mathrm{Mpc}}
\newcommand{\bfi}[1]{\textbf{\textit{#1}}}
\newcommand{\parti}[1]{\frac{\partial #1}{\partial \theta_i}}
\newcommand{\partj}[1]{\frac{\partial #1}{\partial \theta_j}}
\newcommand{\mpc}{{\rm Mpc}}
\newcommand{\eg}{\emph{e.g.}}
\newcommand{\ie}{\emph{i.e.}}

\let\oldAA\AA
\renewcommand{\AA}{\text{\normalfont\oldAA}}
% cmds for this paper 
\newcommand{\gr}{g{-}r}
\newcommand{\fnuv}{FUV{-}NUV}
\newcommand{\sfr}{{\rm SFR}}
\newcommand{\ssfr}{{\rm SSFR}}
\newcommand{\xobs}{{\bf x}_{\rm obs}}

\newcommand{\specialcell}[2][c]{%
  \begin{tabular}[#1]{@{}c@{}}#2\end{tabular}}
% text shih
\newcommand{\foreign}[1]{\textsl{#1}}
\newcommand{\etal}{\foreign{et~al.}}
\newcommand{\opcit}{\foreign{Op.~cit.}}
\newcommand{\documentname}{\textsl{Article}}
\newcommand{\equationname}{equation}
\newcommand{\bitem}{\begin{itemize}}
\newcommand{\eitem}{\end{itemize}}
\newcommand{\beq}{\begin{equation}}
\newcommand{\eeq}{\end{equation}}

%% collaborating
\newcommand{\todo}[1]{\marginpar{\color{red}TODO}{\color{red}#1}}
\definecolor{orange}{rgb}{1,0.5,0}
\newcommand{\chedit}[1]{{\color{orange}#1}}

\begin{document} \sloppy\sloppypar\frenchspacing 

\title{Accelerated Bayesian SED Modeling using Amortized Neural Posterior Estimation}

\newcounter{affilcounter}
\author[0000-0003-1197-0902]{ChangHoon Hahn}
\altaffiliation{changhoon.hahn@princeton.edu.com}
\affil{Department of Astrophysical Sciences, Princeton University, Peyton Hall, Princeton NJ 08544, USA} 

\author{Peter Melchior}
\affil{Department of Astrophysical Sciences, Princeton University, Peyton Hall, Princeton NJ 08544, USA} 

\begin{abstract}
    State-of-the-art spectral energy distribution (SED) analyses use a
    Bayesian framework to infer the physical properties of galaxies from
    observed photometry and spectra.
    They require sampling a high dimensional space of SED model parameters and
    take $>10-100$ CPU hours per galaxy. 
    As a result, they are not computationally feasible for analyzing the {\em
    millions} of galaxy SEDs that will be observed by upcoming galaxy surveys
    (\eg~DESI, PFS, Rubin, James Webb, and Roman). 
    In this work, we present an alternative \emph{scalable} approach for
    conducting rigorous Bayesian inference using Amortized Neural Posterior
    Estimation (ANPE). 
    ANPE is a simulation-based inference method that exploit neural network
    models to estimate the posterior probability distribution over the full
    conditional variable space.
    To demonstrate the ANPE approach, we present {\sc SEDflow}, an SED modeling
    method that combines ANPE with the recent \cite{hahn2022}
    {\sc provabgs} SED model, which uses non-parameteric star formation and
    chemical enrichment histories, to analyze SDSS optical photometry. 
    Once trained, deriving the posterior of galaxy properties with {\sc
    SEDflow} takes \emph{${\sim}1$ second per galaxy}. 
    Futhermore, we validate our method using posteriors derived from standard
    Markov Chain Monte Carlo (MCMC) sampling and mock observations, with known
    true parameter values.  
    The {\sc SEDflow} posteriors accurately estimate the true posteriors of SED
    model parameters and are indistinguishable from MCMC estimates.    
    Therefore, we conclude that using ANPE we derive accurate posteriors
    ${\sim}50,000\times$ faster than standard SED modeling methods. 
    Lastly, we apply our method to 33,887 galaxies in the NASA-Sloan Atlas and
    publicly release the posteriors of galaxy properties for every galaxy.
\end{abstract}
\keywords{galaxies: evolution -- galaxies: statistics}

\section{Introduction} \label{sec:intro} 

\todo{paragraph on SED modeling}

\todo{paragraph on SED Modeling with a full Bayesian approach}
- marginalize over nuisance parameters
- accurately capture degeneracies among model parameters
- allows informative priors based on prior observations

\todo{limitation of Bayesian SED modeling} 
- These methods involve sampling a high dimensional space of SED model
parameters. 
- The dimensionality only increases as SED models become more sophisticated
both in terms of stellar population synthesis modeling and accounting for
observational effects.  
- State-of-the-art methods use MCMC or HMC to efficiently sample a high
dimensional parameter space. Go through all the latest works and detail their
dimensionality and setups. 
- Despite the use of modern sampling techniques, SED modeling take hundreds of
hours per galaxy. Even for $\sim 4000$ galaxies in the LEGA-C ESO Public
Spectroscopic Survey~\cite{}, this required 3.5 million CPU hours {\sc
Bagpipes}. 

\todo{simulation based inference}
provide an overview of simulation based inference 

the Dark Energy Spectroscopic
Survey~\citep[DESI;][]{desicollaboration2016}, the Prime Focus
Spectrograph~\citep[PFS;][]{takada2014}, Rubin~\chedit{CITATION}, and
Roman~\chedit{CITATION}.



\section{Simulation-Based Inference} \label{sec:sbi}
% standard bayesian approach and introducing SBI
The goal of Bayesian SED modeling, and probabilistic inference more
broadly, is to infer the posterior probability distributions
$p(\btheta\given\bfi{x})$ of galaxy properties, $\btheta$, given observations, 
$\bfi{x}$.
For a specific $\btheta$ and $\bfi{x}$, we can evaluate the posterior using
Bayes' rule, 
$p(\btheta\given\bfi{x}) \propto p(\btheta)~p(\bfi{x}\given\btheta)$, where 
$p(\btheta)$ denotes the prior distribution and $p(\bfi{x}\given\btheta)$ the
likelihood, which is typically assumed to have a Gaussian functional form: 
\beq
\label{eq:likelihood}
    \ln p(\bfi{x}\given\btheta) = -\frac{1}{2}\left(\bfi{x} - m(\btheta)\right)^T {\bf C}^{-1}
    \left(\bfi{x} - m(\btheta)\right).
\eeq
$m(\btheta)$ is the theoretical model, in our case a galaxy SED model from SPS.
${\bf C}$ is the covariance matrix of the observations. 
In practice, off-diagonal terms are often ignored and measured uncertainties
are used as estimates of the diagonal terms. 

% overview of SBI and mention of ABC
Simulation-based inference (SBI; also known as ``likelihood-free'' inference)
offers an alternative that requires no assumptions about the form of the
likelihood. 
Instead, SBI uses a generative model, \emph{i.e.} a simulation $F$, to generate
mock data $\bfi{x}'$ given parameters $\btheta'$: $F(\btheta') = \bfi{x}'$. 
It uses a large number of simulated pairs $(\btheta', \bfi{x}')$ to directly estimate
either the posterior  $p(\btheta\given \bfi{x})$, the likelihood
$p(\bfi{x}\given \btheta)$, or the joint distribution of the parameters and data $p(\btheta, \bfi{x})$. 
SBI has already been successfully applied to a number of Bayesian parameter
inference problems in astronomy~\citep[\emph{e.g.}][]{cameron2012, weyant2013,
hahn2017b, kacprzak2018, alsing2018, wong2020, huppenkothen2021, zhang2021}
and in physics~\citep[\emph{e.g.}][]{brehmer2019, cranmer2020}.

%One simple and pedagogical example of SBI is Approximate Bayesian Computation~\citep[ABC;][]{rubin1984, pritchard1999, beaumont2002}, which uses a rejection sampling framework to estimate the posterior.  First, parameter values are sampled from the prior: $\btheta'\sim p(\btheta)$.  The forward model, $F$, is then run on $\btheta'$ to generate simulated data $F(\btheta') = \bfi{x}'$.  If the simulated $\bfi{x}'$ is `close' to the observed $\bfi{x}$, usually based on a threshold on some distance criterion $\rho(\bfi{x}', \bfi{x}) < \epsilon$, $\btheta'$ is kept.  Otherwise, $\btheta'$ is rejected.  This process is repeated until there are enough samples to estimate the posterior.  The estimated posterior from ABC can be written as $p(\btheta \given \rho(F(\btheta), \bfi{x}) < \epsilon)$.  In the case where $\epsilon\rightarrow 0$, the conditional statement is equivalent to the condition $F(\btheta) = \bfi{x}$; thus, the estimated ABC posterior is  equivalent to the true posterior: $p(\btheta \given \rho(F(\btheta), \bfi{x}) < \epsilon\rightarrow 0) \equiv p(\btheta \given \bfi{x})$.


\subsection{Amortized Neural Posterior Estimation} \label{sec:flow}
SBI provides another a critical advantage over MCMC inference methods --- it
enables \emph{amortized inference}. 
For SED modeling using MCMC, each galaxy requires >$10^5$ model evaluations to
accurately estimate $p(\btheta \given \bfi{x})$~(\citealt{hahn2022}, 
Kwon~\etal~in prep.).
Moreover, model evaluations for calculating the posterior of one galaxy cannot
be used for another. 
This makes MCMC approaches for SED modeling of upcoming surveys computationally
infeasible.

With density estimation SBI, we require a large number (${\sim}10^6$) of model
evaluations only initially to train a neural density estimator (NDE), 
\emph{i.e.} a neural network with parameters $\bphi$ that is trained to
estimate the density $p_\phi(\btheta \given \bfi{x}')$.
If the training covers the entire or the practically relevant portions of the
$\btheta$ and $\bfi{x}$ spaces, we can evaluate
$p_\phi(\btheta\given\bfi{x}_i)$ for each galaxy $i$ with minimal computational
cost. 
The inference is therefore amortized and no additional model evaluations are
needed to generate the posterior for each galaxy.
This technique is called  Amortized Neural Posterior Estimation (ANPE) 
and has recently been applied to a broad range of astronomical applications
from analyzing gravitational waves~\citep[\emph{e.g.}][]{wong2020,dax2021} to
binary microlensing lensing~\citep{zhang2021}.
For SED modeling, the choice in favor of using ANPE is easy: the entire upfront
cost for ANPE model evaluations would only yield posteriors of tens of galaxies
with MCMC.

ANPE makes two important assumptions.
First, the simulator $F$ is capable of generating mock data $\bfi{x}'$ that is
practically indistinguishable from the observations.
In terms of the expected signal, $m$ in Eq.~\ref{eq:likelihood}, this is the
same requirement as any probabilistic modeling approach. 
But unlike likelihood-based evaluations, such as conventional MCMC, data
generated for SBI need to include all relevant noise terms as well. 
We address both aspects in Sections \ref{sec:training} and \ref{sec:forward-model}.
Second, ANPE assumes that the NDE is trained well: 
$p_\phi(\btheta \given \bfi{x}')$ is a good approximation of 
$p(\btheta \given \bfi{x}')$, and therefore of $p(\btheta \given \bfi{x})$. 
We assess this in Section~\ref{sec:results}.


ANPE commonly employs so-called ``normalizing flows''~\citep{tabak2010,
tabak2013} as density estimators.
Normalizing flow models use an invertible bijective transformation, $f$, to map
a complex target distribution to a simple base distribution, $\pi(\bfi{z})$, that is
fast to evaluate.
For ANPE, the target distribution is $p(\btheta \given \bfi{x})$ and the
$\pi(\bfi{z})$ is typically a simple multivariate Gaussian, or mixture of Gaussians.
The transformation $f: \bfi{z} \rightarrow \btheta$ must be invertible and have a
tractable Jacobian. 
This is so that we can evaluate the target distribution from $\pi(\bfi{z})$ by
a change of variable:  
\begin{equation} \label{eq:normflow}
    p(\btheta \given {\bf x}) = \pi(\bfi{z}) \Bigl|{\rm det} \left(\frac{\partial
    f^{-1}}{\partial \btheta} \right)\Bigr|.
\end{equation} 
Since the base distribution is easy to evaluate, we can also easily evaluate
the target distribution.  
A neural network is trained to obtain $f$, the collection of its parameters form $\bphi$.
The network typically consists of a series of simple transforms (\emph{e.g.}
shift and scale transforms) that are each invertible and whose Jacobians are
easily calculated. 
By stringing together many such transforms, $f$ provides an extremely flexible
mapping from the base distribution.
%Rather than a single complicated transformation, the network is typically restricted to a series of simple transforms that are each invertible and whose Jacobians are easily calculated. 

Many different normalizing flow models are now available in the
literature~\citep[\emph{e.g.}][]{germain2015, durkan2019}.
In this work, we use Masked Autoregressive
Flow~\citep[MAF;][]{papamakarios2017}. 
The autoregressive design~\citep{uria2016} of MAF is particularly well-suited
for modeling conditional probability distributions such as the posterior. 
Autoregressive models exploit chain rule to expand a joint probability of a set
of random variables as products of one-dimensional conditional
probabilities: $p(\bfi{x}) = \prod_i p(x_i\given x_{1:i-1})$. 
They then use neural networks to describe each conditional probability,
$p(x_i\given x_{1:i-1})$. 
In this context, we can add a conditional variable $y$ on both sides of the
equation, $p(\bfi{x}\given \bfi{y}) = \prod_i p(x_i\given x_{1:i-1}, \bfi{y})$, so that the
autoregressive model describes a conditional probability $p(\bfi{x}\given \bfi{y})$. 
One drawback of autoregressive models is their sensitivity to the ordering of
the variables. 
Masked Autoencoder for Distribution Estimation~\citep[MADE;][]{germain2015}
models address this limitation using binary masks to impose the autoregressive
dependence and by permutating the order of the conditioning variables.
A MAF model is built by stacking multiple MADE models.  
Hence, it has the autoregressive structure of MADE but with more flexibility to
describe complex probability distributions.  
In practice, we use the MAF implementation in the $\mathtt{sbi}$ Python
package\footnote{\url{https://github.com/mackelab/sbi/}}~\citep{greenberg2019,
tejero-cantero2020}.

\section{NASA-Sloan Atlas} \label{sec:obs}
As a demonstration of its speed and accuracy, we apply {\sc SEDflow} to optical
photometry from the NASA-Sloan Atlas\footnote{\url{http://nsatlas.org/}} (NSA)
with some additional quality cuts.
The NSA catalog is a re-reduction of SDSS DR8~\citep{aihara2011} that includes
an improved background subtraction~\citep{blanton2011} and near and far UV
photometry from GALEX~\citep{}.
For optical photometry, we use SDSS photometry in the $u$, $g$, $r$, $i$, and
$z$ bands, which are corrected for galactic extinction using
\cite{schlegel1998}.
For UV photometry, we use GALEX photometry in the $W1$ and $W2$ bands based on
DR6\footnote{\url{http://galex.stsci.edu/GR6/}}.
\todo{details about the GALEX force photometry} 

We impose a number of additional quality cuts to the NSA photometry.
The SDSS photometric pipeline can struggle to accurately define the center
of objects near the edge or at low signal-to-noise. 
In some cases, the centroiding algorithm will report the position of the peak
pixel in a given band as the centroid. 
These cases are often associated with spurious objects, so we exclude them
from our sample. 
We also exclude objects that have pixels, which were not checked for peaks
by the deblender. % ; deblending can be problematic in these cases. 
The SDSS pipeline interpolates over pixels classified as bad (\eg~cosmic ray).
We exclude objects where more than 20\% of point-spread function (PSF) flux is
interpolated over as well as objects where the interpolation affected many
pixels and the PSF flux error is inaccurate. 
We also exclude objects where the interpolated pixels fall within 3 pixels of
their center and they contain a cosmic ray that was interpolated over.
Lastly, we exclude any objects that were not detected at $\ge5\sigma$ in the
original frame, that contain saturated pixels, or where their radial profile 
could not be extracted.
By imposing these quality cuts, we avoid complications from artificats in the
photometry that we do not model. 
In principle, we can relax the cuts if we were to include observational effects
in our model.
For additional details on the quality flags, we refer readers to the SDSS
documentation\footnote{\url{https://www.sdss.org/dr16/algorithms/flags_detail}}.
After the quality cuts, we have 33,887 galaxies in our NSA sample.

%In Figure~\ref{fig:sbi}, we present the distribution of optical and UV magnitudes of the NSA catalog (color). 



%For each such galaxy, we have created image mosaics from SDSS and GALEX and rephotometered the ugriz bands plus the far and near ultraviolet bands in a self-consistent manne0i

% quality cuts that we impose 
% - not peak center: did not use brightest pixel as centroid; hint that an object may not be real
% - not notchecked: object contains pixels which were not checked for peaks by deblender; deblending may be unreliable
% - not PSF_FLUX_INTERP: more than 20% of PSF flux is interpolated over. May cause outliers in color-color plots, e.g.
% - BAD_COUNTS_ERROR: 	interpolation affected many pixels; PSF flux error is inaccurate and likely underestimated.
% - not both INTERP_CENTER and CR: 	interpolated pixel(s) within 3 pix of the center. Photometry may be affected. 	object contains cosmic rays which have been interpolated over; should not affect photometry
% - BINNED1: detected at ≥ 5σ in original imaging frame
% - not SATURATED: contains saturated pixels; affects star-galaxy separation
% - not NOPROFILE : 	only 0 or 1 entries for the radial flux profile; photometric quantities derived from profile are suspect




\section{SEDflow} \label{sec:sedflow}
In this work we apply ANPE to SED modeling of galaxy spectra. 
tl;dr of intro 

\subsection{SED Modeling: PROVABGS} \label{sec:provabgs}
% section explaining specific SED set up 

provabgs set up 

\subsection{Training} \label{sec:training}
paragraph describing the training data
\begin{itemize}
    \item noise model 
\end{itemize}

description of the ANPE training
\begin{itemize}
    \item architecture
    \item validation 
\end{itemize}

\begin{figure}
\begin{center}
\includegraphics[width=0.9\textwidth]{figs/training.pdf}
    \caption{\label{fig:data}
    Joint distribution of SED model parameters ($\log M_*$, $\beta_1$,
    redshift) and photometric magnitudes ($g$, $r$, $z$) for our training set.
    The training set was constructed by sampling parameter values from the
    prior (Table~\ref{tab:prior}), constructing SEDs using a theoretical SPS
    model, and applying our noise model. 
    For details, we refer readers to Section~\ref{sec:sbi_sed}.
    For comparison, we present the distribution of magnitudes for galaxies in
    the NSA catalog (blue). 
    \emph{The training set fully encompasses the observations, thus, our 
    {\sc SEDflow} method can be used to infer the posterior for all NSA
    galaxies}.
    }
\end{center}
\end{figure}



\begin{figure}
\begin{center}
    \includegraphics[width=0.5\textwidth]{figs/ppplot.pdf}
    \caption{\label{fig:pp}
    }
\end{center}
\end{figure}

\begin{figure}
\begin{center}
    \includegraphics[width=0.9\textwidth]{figs/corner.pdf}
    \caption{\label{fig:corner}
    }
\end{center}
\end{figure}


\section{Results} \label{sec:results}
\todo{validate the normaling flow SBI posteriors for a single case}  
compare the corner plot of a posterior derived from MCMC with the SBI 

\todo{validate the derived properties of galaxies for a handful of galaxies}


\todo{paragraph on the computational advantage}

\subsection{Discussion} \label{sec:discuss}
With ANPE, we can generate posteriors for SED modeling >$10^5$ faster than
conventional MCMC-based methods. 
The primarily concern for ANPE is the accuracy of the posteriors.
One of the key factors that determines the accuracy of the ANPE is the training
data.  
To construct our training data, we use simple noise model with a Gaussian
estimate of $p(\sigma_X\given f_X)$ and treat each photometric band separately,
ignoring any covariance (Section~\ref{sec:traiing}). 
The actual $p(\sigma_X\given f_x)$ distrubiton for NSA, as
Figure~\ref{fig:data} illustrates, is not Gaussian. 
There are also significant covariances among the photometry in different bands.
Despite the shortcomings of our training data, our ANPE is \emph{not}
sensitive to the accuracy of our noise model. 
This is because we include $\sigma_X$ as a conditional variable in our ANPE 
(Section~\ref{sec:anpe_train}).
Hence, as long as the observed ${\bf x}_{\rm obs} = (f_X, \sigma_X, z)$ is
sufficiently within the ${\bf x}$-space support of the training data, the ANPE
produces accurate posteriors. 

A more accurate noise model would in principle improve the performance of ANPEs
since the ${\bf x}$-space of the training data will more effectively span the
observations. 
In other words, there will be fewer training data expended in regions of 
${\bf x}$-space that are not occupied by observations.  
However, for our application to SED modeling, we do not find significantly  
better performance when we replace the noise model to a more sophisticated one.
Instead, we find that having a sufficient number of training data is a more
important factor for the ANPE's accuracy. 
\chedit{comment on the number of training data versus accuracy.}

Although the ANPE produces accurate posteriors for ${\bf x}_{\rm obs}$ within
the ${\bf x}$ support of the training data, some objects in the NSA catalog
that are outside of the support (Figure~\ref{fig:obs}). 
For these objects, the ANPE cannot produce accurate posteriors. 
\chedit{
The $g$, $r$, $\sigma_r$ panels of Figure~\ref{fig:data} reveals, for instance,
that there are galaxies outside of the training data in $g-r$ color space.  
Meanwhile $g-\sigma_r$ and $r-\sigma_r$ distribution of observations are within
that of training data. 
This means that the noise model is not the problem. 
}
Some of these objects are likely observational artifacts. 
Even after imposing the selection criteria in Section~\ref{sec:obs}, there
are objects in our samples with problematic photometry.  
\chedit{list a few typical examples of problematic photometry}. 
Since, the SEDs of such artifcats cannot be modeled using an SPS model combined
with noise, they can lie outside of the training ${\bf x}$ support.

% 0. observational artifacts
% 1. noise model is shitty --- this is not the case
% 2. SED model is shitty --- this is definitely partly the case 
Another reason 
But of course that won't be possible if say the SED model is inaccurate and
cannot describe the full data space.  
mention data-driven modeling. 
Expanding the flexiblity and descriptiveness of SED models is beyond the scope
of this work. 
However, we emphasize that limitations of the SED model impacts the
conventional approach with MCMC. 

In Section~\ref{sec:results}
\chedit{assessing accuracy of the posteriors} 
chi2 calculation using best-fit as a first pass. 
accuracy near the edges of the $p(\theta, x)$ space. same infrastructure can be
used to construt a normalizing flow for the likelihood in order to sanity check
all the results 

\todo{extending to higher dimensions}
our sed models have a lot of dimensions but it's actually not enough. 
we don't have sed model parameters that account for uncertainties in stellar
evolution, IMF, etc. 
They should also include more nuisance parameters to deal with uncertainties in
the observations. 
Go through some of the caveats in \url{http://nsatlas.org/caveats}. 
% something about zero-point calibration by Rachel during the discussion
As dimensionality increases, current methods will get worse and worse. 
Fortunately, conditional normalizing flows have been shown to accurately
estimate far higher dimensional distributions (cite some examples from machine
learning). 

\todo{extra advantages of faster posteriors}
reemphasize that we can meet the needs of DESI, PFS, Rubin, JWST, and Roman. 

recently works have demonstrated that model priors play an exigent role in the
derived galaxy properties. 
Even ``uniformative'' uniform priors on SED model parameters can impose
undesirable priors on derived galaxy properties such as $M_*$, SFR, SFH, or
ZH.
This underscores the importance of carefully selecting priors and validating
results using multiple different priors. 
Our SBI-based approach, can easily include different priors without
reevaluating the training set.
For a different prior, we can adjust the training set so that the set of SED
model parameters $p(\theta^{\rm train})$ follows the prior distribution. 
The CNF can be re-trained and deployed to rapidly reanalyze the entire dataset
in a matter of hours. 

\todo{next steps: extending to spectroscopy}
%In Hahn (in prep), I demonstrate that priors on the SED model parameters can be adjusted to impose uniform priors on derived properties based on the maximum-entorpy method from \cite{handley2019}. 

\section{Summary} \label{sec:summary}


\section*{Acknowledgements}
It's a pleasure to thank 
    Adam Carnall, 
    Miles Cranmer, 
    Kartheik Iyer,
    Andy Goulding,
    Jenny E. Green,
    Uro{\u s}~Seljak,
    Michael A. Strauss, 
    ...
for valuable discussions and comments.

\appendix
%\bibliographystyle{mnras}
\bibliography{sedflow} 
\end{document}
