\section{NASA-Sloan Atlas} \label{sec:obs}
As a demonstration of its speed and accuracy, we apply {\sc SEDflow} to optical
photometry from the NASA-Sloan Atlas\footnote{\url{http://nsatlas.org/}} (NSA)
with some additional quality cuts.
The NSA catalog is a re-reduction of SDSS DR8~\citep{aihara2011} that includes
an improved background subtraction~\citep{blanton2011} and near and far UV
photometry from GALEX~\citep{}.
For optical photometry, we use SDSS photometry in the $u$, $g$, $r$, $i$, and
$z$ bands, which are corrected for galactic extinction using
\cite{schlegel1998}.
For UV photometry, we use GALEX photometry in the $W1$ and $W2$ bands based on
DR6\footnote{\url{http://galex.stsci.edu/GR6/}}.
\todo{details about the GALEX force photometry} 

We impose a number of additional quality cuts to the NSA photometry.
The SDSS photometric pipeline can struggle to accurately define the center
of objects near the edge or at low signal-to-noise. 
In some cases, the centroiding algorithm will report the position of the peak
pixel in a given band as the centroid. 
These cases are often associated with spurious objects, so we exclude them
from our sample. 
We also exclude objects that have pixels, which were not checked for peaks
by the deblender. % ; deblending can be problematic in these cases. 
The SDSS pipeline interpolates over pixels classified as bad (\eg~cosmic ray).
We exclude objects where more than 20\% of point-spread function (PSF) flux is
interpolated over as well as objects where the interpolation affected many
pixels and the PSF flux error is inaccurate. 
We also exclude objects where the interpolated pixels fall within 3 pixels of
their center and they contain a cosmic ray that was interpolated over.
Lastly, we exclude any objects that were not detected at $\ge5\sigma$ in the
original frame, that contain saturated pixels, or where their radial profile 
could not be extracted.
By imposing these quality cuts, we avoid complications from artificats in the
photometry that we do not model. 
In principle, we can relax the cuts if we were to include observational effects
in our model.
For additional details on the quality flags, we refer readers to the SDSS
documentation\footnote{\url{https://www.sdss.org/dr16/algorithms/flags_detail}}.
After the quality cuts, we have 33,887 galaxies in our NSA sample.

In Figure~\ref{fig:sbi}, we present the distribution of optical and UV
magnitudes of the NSA catalog (color). 



%For each such galaxy, we have created image mosaics from SDSS and GALEX and rephotometered the ugriz bands plus the far and near ultraviolet bands in a self-consistent manne0i

% quality cuts that we impose 
% - not peak center: did not use brightest pixel as centroid; hint that an object may not be real
% - not notchecked: object contains pixels which were not checked for peaks by deblender; deblending may be unreliable
% - not PSF_FLUX_INTERP: more than 20% of PSF flux is interpolated over. May cause outliers in color-color plots, e.g.
% - BAD_COUNTS_ERROR: 	interpolation affected many pixels; PSF flux error is inaccurate and likely underestimated.
% - not both INTERP_CENTER and CR: 	interpolated pixel(s) within 3 pix of the center. Photometry may be affected. 	object contains cosmic rays which have been interpolated over; should not affect photometry
% - BINNED1: detected at ≥ 5σ in original imaging frame
% - not SATURATED: contains saturated pixels; affects star-galaxy separation
% - not NOPROFILE : 	only 0 or 1 entries for the radial flux profile; photometric quantities derived from profile are suspect



