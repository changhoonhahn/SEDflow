\subsection{Discussion} \label{sec:discuss}

\chedit{discuss the accuracy requirements of the noise model}
Overall, we use a crude Gaussian estimate of $p(\sigma_X | f_X)$ and treat each
photometric band separately, ignoring any covariance. 
In our ANPE, $\sigma_X$ is included in the conditional statement. 
As a result, the accuracy of the posteriors from our ANPE is \emph{not}
sensitive to accurately sampling $\sigma_X$ as long as the assigned $\sigma'_X$
of the training data spans the observations.  

\chedit{what happens if the data}
extends past the $p(\theta, x)$ space.
in principle we could train it well beyond the $p(\theta, x)$ space so that
there's support everywhere. 
But of course that won't be possible if say the SED model is inaccurate and
cannot describe the full data space.  
mention data-driven modeling. 
Expanding the flexiblity and descriptiveness of SED models is beyond the scope
of this work. 
However, we emphasize that limitations of the SED model impacts the
conventional approach with MCMC. 

\chedit{assessing accuracy of the posteriors} 
chi2 calculation using best-fit as a first pass. 
accuracy near the edges of the $p(\theta, x)$ space. same infrastructure can be
used to construt a normalizing flow for the likelihood in order to sanity check
all the results 

\todo{extending to higher dimensions}
our sed models have a lot of dimensions but it's actually not enough. 
we don't have sed model parameters that account for uncertainties in stellar
evolution, IMF, etc. 
They should also include more nuisance parameters to deal with uncertainties in
the observations. 
Go through some of the caveats in \url{http://nsatlas.org/caveats}. 
% something about zero-point calibration by Rachel during the discussion
As dimensionality increases, current methods will get worse and worse. 
Fortunately, conditional normalizing flows have been shown to accurately
estimate far higher dimensional distributions (cite some examples from machine
learning). 

\todo{extra advantages of faster posteriors}
reemphasize that we can meet the needs of DESI, PFS, Rubin, JWST, and Roman. 

recently works have demonstrated that model priors play an exigent role in the
derived galaxy properties. 
Even ``uniformative'' uniform priors on SED model parameters can impose
undesirable priors on derived galaxy properties such as $M_*$, SFR, SFH, or
ZH.
This underscores the importance of carefully selecting priors and validating
results using multiple different priors. 
Our SBI-based approach, can easily include different priors without
reevaluating the training set.
For a different prior, we can adjust the training set so that the set of SED
model parameters $p(\theta^{\rm train})$ follows the prior distribution. 
The CNF can be re-trained and deployed to rapidly reanalyze the entire dataset
in a matter of hours. 

%In Hahn (in prep), I demonstrate that priors on the SED model parameters can be adjusted to impose uniform priors on derived properties based on the maximum-entorpy method from \cite{handley2019}. 

